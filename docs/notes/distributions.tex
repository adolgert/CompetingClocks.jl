\documentclass{article}
\usepackage{hyperref}

\title{Sampling Distributions in Julia}
\author{Andrew Dolgert}
\begin{document}
\maketitle

\section{Introduction}

In order to write a continuous-time simulation, we need to sample distributions. There are many ways to sample these distributions, and they involve different mathematical manipulations of those distributions. This document looks at the functions defined by Julia for manipulating distributions, in order to figure out how to write samplers.


\section{Notation for Distributions}
We need to use some accepted notation for a statistical distribution. Let's choose some variable names to use.

A \emph{cumulative distribution} is the probability of an event before a given time. The random variate is $T$ and $t$ is a parameter. $P$ denotes a probability, and $F$ is the name we choose for a general cumulative distribution.
\begin{equation}
  P(T \le t) = F(t)
\end{equation}
The survival is the probability an event fires after time $t$.
\begin{equation}
  P(T > t) = G(t) = 1 - F(t)
\end{equation}
The \emph{probability density function,} or pdf, is the derivative of the cumulative distribution.
\begin{equation}
  f(t) = dF(t)/dt
\end{equation}
% This distribution can combine continuous and point distributions.
% \begin{equation}
%   F(t) = F'(t) + \sum_i f_i\delta(t-t_i)
% \end{equation}
The \emph{hazard rate} is the probability per unit time of an event, given that it has not yet fired.
\begin{equation}
\lim_{\delta t\rightarrow 0} P(t < T\le t + \delta t) = \lambda(t)
\end{equation}
Every continuous distribution can be written in terms of the hazard rate.
\begin{equation}
F(t) = 1 - e^{-\int_0^t\lambda(s)ds}
\end{equation}
This means the pdf is also a function of hazard rate.
\begin{equation}
f(t) = \lambda(t)e^{-\int_0^t\lambda(s)ds}
\end{equation}
The survival, in terms of the hazard rate, is one term.
\begin{equation}
G(t)=e^{-\int_0^t\lambda(s)ds}
\end{equation}
The log of the survival is called the integrated hazard.
\begin{equation}
\ln G(t) = -\int_0^t\lambda(s)ds = -\Lambda(t)
\end{equation}

\section{Julia Distributions}
Julia has a \texttt{Distributions.jl} library that contains univariate distributions, which are what we want. Let's walk through the functions this package offers for working with those distributions. We can write each function's mathematical equivalent.

\subsection{Parameters}
Julia has one function to retrieve all parameters of a distribution and a few functions that are specific to common parameters.

\begin{itemize}
	\item \texttt{params(d::UnivariateDistribution)}---Gets all parameters for a distribution.
	\item \texttt{scale(d::UnivariateDistribution)}---If the distribution has a parameter $\theta$ and is written as $f(t/\theta)$, then $\theta$ is the scale.
	\item \texttt{rate(d::UnivariateDistribution)}---If the distribution has a parameter $\beta$ and is written as $f(\beta t)$, then $\beta$ is the rate. We see that $\beta\theta = 1$.
	\item \texttt{location(d::UnivariateDistribution)}---If the distribution has a parameter $\mu$ and is written as $f(t-\mu)$, then $\mu$ is the location. This will be important for re-sampling distributions which failed to fire.
	\item \texttt{shape(d::UnivariateDistribution)}---The shape is often a power, $k$, of $t^k$ in the pdf.
\end{itemize}

\subsection{Probability Evaluation}

\begin{table}
\begin{tabular}{lll}
	Julia Function & Equation & Statistics name \\ \hline
	\texttt{ccdf} & $S(x) = e^{-\int^x\lambda(s)ds}$ & Survival \\
	\texttt{cdf} & $F(x) = 1 - S(x) = 1-e^{-\int^x\lambda(s)ds}$ & Cumulative distribution function \\
	\texttt{pdf} & $f(x) = \lambda(x)e^{-\int^x\lambda(s)ds}$ & Probability distribution function \\
	\texttt{logpdf} & $\ln\:f(x)$ & Log-likelihood \\
	\texttt{logcdf} & $\ln\:F(x)$ & \\
	\texttt{logdiffcdf} & $\ln\left(F(x_2)-F(x_1)\right)$ & \\
	\texttt{logccdf} & $-\int^x\lambda(s)ds$ & Integrated hazard (negated) \\
	\texttt{quantile} & $y = F^{-1}(x)$ & Inverse cumulative distribution function \\
	\texttt{cquantile} & $y = F^{-1}(1-x)$ so $x = S(y)$ & Inverse Survival \\
	\texttt{invlogcdf} & $x = \ln\:F(y)$ so $e^x = F(y)$ & \\
	\texttt{invlogccdf} & $x = -\int_0^y\lambda(s)ds$ & Inverse integrated hazard
\end{tabular}
\caption{This translates between Julia functions and hazard-based notation. We can use this to find
the shortest path to our calculation in code.\label{julia-translation}}
\end{table}

\texttt{insupport(d::UnivariateDistribution, x::Any)}---Whether $t$ is in the domain of $f(t)$.

\texttt{pdf(d::UnivariateDistribution, x::Real)}---This is $f(t)$.

\texttt{logpdf(d::UnivariateDistribution, x::Real)}---It's $\ln(f(t))$. This is important because the pdf is a likelihood, so you get the log-likelihood.

\texttt{cdf(d::UnivariateDistribution, x::Real)}---$F(t)$.

\texttt{logcdf(d::UnivariateDistribution, x::Real)}---$\ln(F(t))$.

\texttt{logdiffcdf(d::UnivariateDistribution, x::Real, y::Real)}---$\ln(F(x))-\ln(F(y))$.

\texttt{ccdf(d::UnivariateDistribution, x::Real)}---Survival, $G(t) = 1-F(t)$.

\texttt{logccdf(d::UnivariateDistribution, x::Real)}---The log of the survival, which is called the integrated hazard, $\Lambda(t)$.

\texttt{invlogcdf(d::UnivariateDistribution, x::Real)}---The inverse function of log-CDF. That's
\begin{equation}
x = \ln F(y)
\end{equation}
so
\begin{equation}
e^x = F(y) = 1 - e^{\int_0^y\lambda(s)ds}.
\end{equation}

\texttt{invlogccdf(d::UnivariateDistribution, x::Real)}---This inverse is one I've used.
\begin{equation}
e^x = G(y) = e^{\int_0^y\lambda(s)ds}
\end{equation}
That means this function is the inverse of the integrated hazard.
\begin{equation}
x = \Lambda(y)
\end{equation}


\section{Next Reaction for Non-Markov Processes}

Gibson and Bruck describe the Next Reaction Method for non-Markov processes by focusing on one step of the algorithm. Assuming there was a draw for some time $\tau$ when the distribution $F_{a_n}$ would fire, and there is now a change to create distribution $F_{a,n+1}$, the new draw should be
\begin{equation}
	\tau'=F_{a,n+1}^{-1}\left(\frac{F_{a,n}(\tau) - F_{a,n}(t_n)}{1-F_{a,n}(t_n)}\right).
\end{equation}
Shifting to get rid of the inverse, this becomes
\begin{equation}
	F_{a,n+1}(\tau')=\frac{F_{a,n}(\tau) - F_{a,n}(t_n)}{1-F_{a,n}(t_n)}.
\end{equation}
I suspect this equation will make a lot of sense if we write it in terms of survivals. Start by writing it as $S=1-F$
\begin{equation}
	1-F_{a,n+1}(\tau')=\frac{1-F_{a,n}(\tau)}{1-F_{a,n}(t_n)}.
\end{equation}
Multiplying both sides by the denonminator shows a known rule about survival functions. The rule is that \emph{conditional survival is multiplicative.}
\begin{equation}
	\left(1-F_{a,n}(t_n)\right)\left(1-F_{a,n+1}(\tau')\right)=1-F_{a,n}(\tau).\label{eqn:survivalrule}
\end{equation}
The equation says that we expect the proposed survival of this process to time $\tau$ to be the same as the survival to from start to $t_n$ and the survival from $t_n$ to $\tau'$.

Have you been anxious about the notation? The Gibson and Bruck paper failed with the notation, and we just copied it. The problem is how they annotated the zero-time for distributions. We can clarify it by writing the equation in terms of hazards.
\begin{equation}
	\exp\left(-\int_{t_0}^{\tau}\lambda_n(s)ds\right) = \exp\left(-\int_{t_0}^{t_n}\lambda_n(s)ds\right)\exp\left(-\int_{t_n}^{\tau'}\lambda_{n+1}(s)ds\right)
\end{equation}
The integrals cover the relevant durations and the hazards are the hazards of those distributions during those durations. This equation also leads to an observation. Anderson's method is the Next Reaction method in log space. Who knew? Everybody. Everybody knew.

We can make our notation in a way that will have an echo in the code. All software libraries treat distributions as starting at time zero and leave it to the client to shift their absolute time. Therefore, let's think about distributions as an enabling time that defines the zero, a current time, and the distribution, $(t_e, t_n, f)$.

If we think of the first draw as a draw by inversion, we can write it as inversion of a conditional survival.
\begin{equation}
	U = S(t_n - t_e, \tau - t_n)
\end{equation}
Because $0<U<1$, this could equally well be $U=F(t_n-t_e, \tau-t_n)$ equally well. If the enabling time is in the future, that's fine, but it changes the draw to use a marginal survival.
\begin{equation}
	U = S(0, \tau - t_e) = S(\tau - t_e)
\end{equation}
This $U$ is the right-hand side of \ref{eqn:survivalrule}.

When Gibson and Bruck discuss an update rule, it requires storing the remaining $U$.
\begin{enumerate}
	\item When we make the first draw, we save $U$.
	\item Each time, $t_n$, a transition is disabled, we calculate
		\begin{equation}
			U'=\frac{U}{S(t_n-t_e, \tau - t_n)}
		\end{equation}
	\item When the transition is enabled or changed, we calculate its draw by inversion with the saved $U'$,
	\begin{equation}
		U'=S(t_n-t_e, \tau' - t_n).
	\end{equation}
\end{enumerate}
While the first draw of $\tau$ can be done using appropriate, non-inversion, methods for any distribution, it is a weakness of this method that later enabling of the transition requires using inversion, which can be much slower and more error-prone.

Let's walk through the cases because they clarify what we need to store within the sampler.
\begin{itemize}
	\item First time enabling the transition. Sample any way possible from $S(t_n-t_e, \tau-t_n)$, but then calculate and store $U=S(t_n-t_e, \tau-t_n)$.
	\item Enabling a transition that fired. Sample any way possible from $S(t_n-t_e, \tau-t_n)$, but then calculate and store $U=S(t_n-t_e, \tau-t_n)$.
	\item Enabling a transition that was disabled before it fired.
	\begin{itemize}
		\item General case. Sample by inversion $U=S(t_n-t_e, \tau-t_n)$.
		\item The transition is picking up where it left off. We don't have to sample because the firing time is just pushed forward. Does this happen often enough to care? If the previous distribution was $(t_e, t_n, f)$ when it was disabled at time $t_n$, we know the transition is merely shifted because, at time $t_{n+1}$, it has the form $(t_{n+1}-(t_e-t_n), t_{n+1}, f)$. In this case, $\tau' = \tau + t_{n+1}-t_n$, and there is no change to the stored $U$.
	\end{itemize}
	\item Disabling a transition that did not fire. If a transition did not fire at time $t_{n+1}$, calculate conditional survival, $S(t_n-t_e, t_{n+1}-t_e)$ and adjust the stored $U$ by dividing by the conditional survival.
	\item Disabling the transition that fired. If we adjust the conditional survival according to the formula above, it may close to zero but not equal. We need it to be $U=0$, so set that value.
\end{itemize}
From the cases above, the sampler needs to store two kinds of information.
\begin{itemize}
	\item Long-term values about all transitions. These are $U$, $t_e-t_n$, and $f$.
	\item Short-term values for currently-enabled transitions. These are $t_e$, $t_n$, $f$.
\end{itemize}

There are a couple of challenges. One is how to know when two floating-point values are equal, such as in the check that the transition is picking up where it left off. This is handled in code by calculating the machine epsilon for a floating point of the size of $t_n$. Multiply that by a small factor, such as 2 or 4, and use that as an error bound.

Another challenge is how to figure out which disabled transition is the one that fired. If the caller specifies it is disabling the transition that fired, then it's clear. If the caller doesn't specify, can we know? The transitions are in a heap where the soonest is on top. If the first transition that's disabled was the one on top and the next transition becomes the one on top, it could look like it was next to fire. We could check the times on those transitions in order to see whether the time of disabling matches the time of the top transition, but that's prone to error because it requires equality of floating-point numbers. There will definitely be some time when two numbers in a continuous time simulation are very close to each other.

On the other hand, there isn't a good way to use a Next Reaction sampler except to always choose the transition it sampled to be next. Let's store the response from the next() function and assume that's the one that fired when it gets disabled.

\end{document}
